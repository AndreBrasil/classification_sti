This work presents a comparative study between the Brazilian and Dutch national evaluation systems of research and graduate education. Science systems can be as distinct as the social-economic circumstances, established governance, and cultural realities of each country, and both evaluation systems under analysis were developed from very different conceptions of assessment, university autonomy, and governance of higher education, science, and technology. From the study of policy and guiding documents, connected legislation, and related literature, we compare the design of the science system in each country, examining their impact on the adopted evaluation models. For that, we adapt established comparison frameworks and focus the analyses on influential aspects for each system, such as the links between evaluation and funding, or the consequential effect the results can have on researchers' behaviour. While Brazil has lately sought inspiration from the long-standing and stable Dutch evaluation -- which is recognised as a critical factor in the country's quality assurance -- the Latin American country has already developed one of the most sophisticated performance-based evaluation systems worldwide. Thus, we conclude by highlighting inspiring methods and approaches from each evaluation system so that those lessons could lead to positive change for both countries. 