
{\footnotesize \renewcommand{\arraystretch}{1.3} \linespread{0.8}
\begin{xltabular}{\linewidth}{@{}L{3.8cm}XX@{}}
\caption{Main characteristics of the Brazilian and Dutch evaluation systems}\label{tab:compeval:compare}\\ 

\toprule
\textbf{Categories} & \textbf{Brazil} & \textbf{The Netherlands}  \\\midrule

\endfirsthead

\Cref*{tab:compeval:compare} Continued\\ 
\toprule
 \textbf{Categories} & \textbf{Brazil} & \textbf{The Netherlands} \\ \midrule

\endhead

\bottomrule
\multicolumn{3}{r@{}}{Continue\ldots}\\ 
\endfoot

\bottomrule
\multicolumn{3}{c@{}}{Source: \textcite{Brasil.2022e, Scholten.2018, VSNU.2020}.}
\endlastfoot

\textbf{Name} & Quadrennial Evaluation & Strategy Evaluation Protocol \\
\textbf{Level} &	National & Institutional\\
\textbf{Responsible entity} & National agency & Institutions\\
\textbf{Legal framework}	& Legal ordinances and field-specific documents (both periodically issued) & Higher Education Act (WHW) enforced through the Strategy Evaluation Protocol (SEP)  \\ 
\textbf{Unit of assessment} & Graduate program & Research unit \\
\textbf{Time framework} & 4 years & 6 years\\
\textbf{Method} & Informed peer review & Informed Peer Review \\
\textbf{Evaluation type} & Performance-based	& Formative\\
\textbf{Accreditation effects} & Results determine if accreditation of graduate program is renewed	& Results do not impact accreditation\\
\textbf{SSH specificity} & Evaluated field-specifically & Department-based evaluation, with a separate addendum for the Humanities\\
\textbf{Bibliometric data} & Web of Science/Scopus/Google Scholar/CRIS &	Each research unit selects data sources (and indicators) that better support their self-assessment\\
\textbf{Scientometric data} & Comprehensive data collection is conducted yearly from every graduate program (microdata level) & A custom-made selection of sources is made by each research unit, according to their self-assessment approach\\
\textbf{Benchmarking} & Evaluation is comparative at a national scale, but within 49 evaluation areas. & Research units usually decide whether they want to include a benchmark and which other units they would include. In some fields, a national benchmarking is also possible (e.g., psychology)  \\
\textbf{Site visit}	& It may be recommended by the assessment committees in special circumstances. & Integral part of the process, being mandatory in the current evaluation protocol.\\
\textbf{Language preference} & National language & Push to English\\
\textbf{Funding link} & Strong & Weak\\
\textbf{Changes over time} & Varies & Few\\
\textbf{Transparency} & Strong & Weak \\
\textbf{Controversies} & Varies & Few \\
\textbf{Rapporteur’s perception of influence on researchers’ way of working} & Strong & Weak\\

\end{xltabular}
}