
{\footnotesize \renewcommand{\arraystretch}{1.3} \linespread{0.8}
\begin{xltabular}{\linewidth}{@{}L{3.8cm}XX@{}}
\caption{Methodological comparison of the Brazilian and Dutch evaluation systems}\label{tab:compeval:methods}\\ 

\toprule
\textbf{Categories} & \textbf{Brazil} & \textbf{The Netherlands}  \\\midrule

\endfirsthead

\Cref*{tab:compeval:methods} Continued\\ 
\toprule
 \textbf{Categories} & \textbf{Brazil} & \textbf{The Netherlands} \\ \midrule

\endhead

\bottomrule
\multicolumn{3}{r@{}}{Continue\ldots}\\ 
\endfoot

\bottomrule
\multicolumn{3}{@{}c@{}}{Source: \textcite{Brasil.2022e},  \AtNextCitekey{\defcounter{maxnames}{1}}\textcite{Scholten.2018, VSNU.2020}.}
\endlastfoot

\textbf{Method} & Informed peer review & Informed peer review\\
\textbf{Language preference} & National language & Push to English\\
\textbf{Bibliometric data} & Web of Science/Scopus/Google Scholar/CRIS &	Each research unit selects data sources (and indicators) that better support their self-assessment\\
\textbf{Scientometric data} & Comprehensive data collection is conducted yearly from every graduate program (microdata level) & A custom-made selection of sources is made by each research unit, according to their self-assessment approach\\
\textbf{Benchmarking} & Evaluation is comparative at a national scale, but within 49 evaluation areas. & Research units usually decide whether they want to include a benchmark and which other units they would include. In some fields, national benchmarking is also possible (e.g., psychology)  \\
\textbf{SSH specificity} & Evaluated field-specifically & Department-based evaluation, with a separate addendum for the Humanities\\
\textbf{Site visit}	& It may be recommended by the assessment committees in special circumstances. & Integral part of the process, mandatory in the current evaluation protocol.\\

\end{xltabular}
}